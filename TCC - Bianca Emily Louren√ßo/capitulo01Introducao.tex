\chapter{Introdução}
\label{cap:01}

O objetivo deste documento é esclarecer aos autores o formato que deve ser utilizado nos relatórios técnicos a serem submetidos ao final dos cursos de Graduação e Pós-Graduação do IFSP câmpus São João da Boa Vista. Este documento está escrito de acordo com o modelo indicado para a formatação dos relatórios técnicos; assim, serve de referência, ao mesmo tempo em que comenta os diversos aspectos da formatação.

Observe as instruções e formate seu relatório técnico de acordo com este padrão. Lembre-se que uma formatação correta contribui para uma boa avaliação do seu trabalho.

Além disso, neste documento estão listadas as seções obrigatórias que você deverá fornecer, bem como os exemplos dos comandos mais comuns que serão utilizados na construção de seu documento. Para pesquisar sobre mais comandos, recomenda-se a utilização do site \url{https://ctan.org/}, que é a biblioteca principal do \LaTeX, e o do site \url{https://tex.stackexchange.com} que é uma das principais comunidades para solução de dúvidas relacionadas a \LaTeX. Ambas são em inglês.

A introdução é um elemento preliminar, opcional, utilizado para fornecer informações específicas, comentar tecnicamente o conteúdo do trabalho, além de evidenciar as motivações que levaram o autor à escolha de determinado tema.

Trata-se de importante estratégia de aproximação, pois permite valorizar a escolha do assunto, mostrar a relevância da abordagem temática e esclarecer quanto ao passo-a-passo utilizado na estruturação do texto.

Na introdução, o leitor terá condições de avaliar:

\begin{itemize}
	\item O grau de informação, conhecimento e competência técnica do autor relativamente ao assunto a ser tratado;
	\item A qualidade, a eficiência, a originalidade e o ineditismo de sua abordagem;
	\item A pertinência das informações apresentadas e a possibilidade de acrescentar algo de novo ao universo conceitual do leitor.
\end{itemize}


\section{Objetivos}

\subsection{Objetivo Geral}

Desenvolver um \textit{cozy game} 2D utilizando a \textit{engine Godot}, com foco em oferecer uma experiência relaxante e imersiva por meio de mecânicas simples, estética acolhedora e narrativa envolvente.

\subsection{Objetivos Específicos}
\begin{itemize}
	\item Definir o estilo visual e narrativo do jogo;
	\item Projetar jogabilidade e interações relaxantes;
	\item Preparar o ambiente de desenvolvimento na \textit{Godot};
	\item Criar ou adquirir os \textit{assets} visuais e sonoros;
	\item Programar as mecânicas e sistemas do jogo.
	
\end{itemize}